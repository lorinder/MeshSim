\documentclass{article}

\title{Configuration}

\begin{document}

\section{Topology overview}

The Backhaul node has IP address 10.1.1.1.  In the Mesh, the interfaces
communicating among each other have IP addresses 10.1.2.$x$; the Mesh
access points (APs) have IP addresses 10.1.3.$x$ with $1 \leq x \leq
100$, the stations (STAs) have IP addresses 10.1.3.$y$ with $101 \leq
y$.

\section{Configuration}

\subsection{Command line arguments}

The most basic settings (e.g., the number of mesh nodes, number of STAs,
etc.) are currently configured on the command line, run
\begin{verbatim}
mesh_sim --help
\end{verbatim}
for a list of parameters that are set this way.

Additionally, as a first positional argument, \verb@mesh_sim@ takes the
name of a directory where configuration files reside.  The second
positional argument is the location of the output directory where pcap
files will be placed.

\subsection{Configuration files}

Configuration files have an ad-hoc format which is geared towards easy
readability and editability for the task at hand.  Lines starting with \#
is are comment lines and skipped.  Empty lines are skipped as well.
The other lines typically have several tokens separated by white space.

There are multiple configuration files in a configuration directory:
\begin{itemize}

\item \verb@mesh_mobility.txt@.  This file contains information about
the locations of mesh nodes.

\item \verb@sta_mobility.txt@.  This file contains information about the
location of STAs.

\item \verb@apps.txt@.  This file contains information about which apps
run on which nodes.

\end{itemize}
There is a set of well documented sample configuration files in the
\verb@examples/conf/@ directory.  Please refer to them for further
details.  The sample files also work, run them with
\begin{verbatim}
./build/sim/mesh_sim examples/conf
\end{verbatim}
from the root of the tree.  (This assumes, the build directory was
placed at \verb@build/@.)

\end{document}
